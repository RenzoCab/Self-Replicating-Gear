\documentclass[12pt]{article}
\usepackage[table]{xcolor}
\usepackage[margin=1in]{geometry} 
\usepackage{amsmath,amsthm,amssymb}
\usepackage[english]{babel}
\usepackage{tcolorbox}
\usepackage{enumitem}
\usepackage{hyperref}
\usepackage{listings}
\usepackage{blkarray}
\usepackage{float}
\usepackage{bm}
\usepackage{subfigure}
\usepackage{booktabs}
\usepackage{siunitx}
\usepackage{commath}

\setcounter{secnumdepth}{5}
\setcounter{tocdepth}{5}

\newtheorem{theorem}{Theorem}[section]
\newtheorem{corollary}{Corollary}[theorem]
\newtheorem{lemma}[theorem]{Lemma}
\newtheorem{proposition}[theorem]{proposition}
\newtheorem{exmp}{Example}[section]\newtheorem{definition}{Definition}[section]
\newtheorem{remark}{Remark}
\newtheorem{ex}{Exercise}
\theoremstyle{definition}
\theoremstyle{remark}
\bibliographystyle{elsarticle-num}

\DeclareMathOperator{\sinc}{sinc}
\newcommand{\RNum}[1]{\uppercase\expandafter{\romannumeral #1\relax}}
\newcommand{\N}{\mathbb{N}}
\newcommand{\Z}{\mathbb{Z}}
\newcommand{\R}{\mathbb{R}}
\newcommand{\E}{\mathbb{E}}
\newcommand{\matindex}[1]{\mbox{\scriptsize#1}}
\newcommand{\V}{\mathbb{V}}
\newcommand{\Q}{\mathbb{Q}}
\newcommand{\K}{\mathbb{K}}
\newcommand{\C}{\mathbb{C}}
\newcommand{\prob}{\mathbb{P}}

\lstset{numbers=left, numberstyle=\tiny, stepnumber=1, numbersep=5pt}

%\pagenumbering{gobble}
\setcounter{secnumdepth}{0} % This line removes the numbers in the titles.

\begin{document}
\title{Can a printer print itself anew while being worn out?}
\author{Renzo Miguel Caballero Rosas\\
\url{Renzo.CaballeroRosas@kaust.edu.sa}} 
%\date{\vspace{-5ex}}
\maketitle

%\pagebreak
\tableofcontents
%\pagebreak

\section{Introduction}

Self-repairing robots are taking more importance with time; as a first approximation to this field, we use a robot which specialty is creating new parts, a 3D printer.

\section{Preliminary Steps}

\subsection{Communication Channel}

We start the work focusing on the communication PC-Printer. We are willing to use serial communication as suggested by some knowledgeable group members. We will first establish communication between well-known and controllable devices (PC-Arduino Nano), and after we can jump to the communication PC-Printer.\\
\quad\\
Linux is not a real time operating system. So, we cannot generate pulses with the required timings to control stepper motors directly from the board pins with running software, even as a kernel module. Then, how can we use steppers and high-level Linux features? (\url{https://medium.com/iotforall/how-to-build-a-3d-printer-in-python-b05af32489f5})\\
One solution is PyCNC, but it needs additional hardware to control the motors. Our purpose is to use the actual printer's hardware and send instructions by Python.

\end{document}