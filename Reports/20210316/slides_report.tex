\PassOptionsToPackage{table}{xcolor}
\documentclass[aspectratio=169]{beamer}\usepackage[utf8]{inputenc}
\usepackage{lmodern}
\usepackage[english]{babel}
\usepackage{color}
\usepackage{amsmath,mathtools}
\usepackage{booktabs}
\usepackage{mathptmx}
\usepackage[11pt]{moresize}
\usepackage{hyperref}
\usepackage{commath}
\usepackage{bm}
\usepackage{subfigure}
\usepackage{siunitx}
\usepackage{amsthm}

\setbeamertemplate{navigation symbols}{}
\setbeamersize{text margin left=5mm,text margin right=5mm}
\setbeamertemplate{caption}[numbered]
\addtobeamertemplate{navigation symbols}{}{
\usebeamerfont{footline}
\usebeamercolor[fg]{footline}
\hspace{1em}
\insertframenumber/\inserttotalframenumber}

\newcommand{\R}{\mathbb{R}}
\newcommand{\E}{\mathbb{E}}
\newcommand{\N}{\mathbb{N}}
\newcommand{\Z}{\mathbb{Z}}
\newcommand{\V}{\mathbb{V}}
\newcommand{\Q}{\mathbb{Q}}
\newcommand{\K}{\mathbb{K}}
\newcommand{\C}{\mathbb{C}}
\newcommand{\T}{\mathbb{T}}
\newcommand{\I}{\mathbb{I}}

\title{3D Body Metric}
\subtitle{Renzo Caballero}

\begin{document}

\begin{frame}
\titlepage
\end{frame}


\setbeamercolor{background canvas}{bg=white!10}
\begin{frame}\frametitle{Motivation}

Let $A,B\subset\R^3$ two sets describing 3D bodies.\\
\quad\\
We want a mathematical way to describe when both bodies are equal. As an example, let $A=B_{([0,0,0],1)}$ and $B=B_{([1,1,1],1)}$ the unitary balls with centers in $(0,0,0)\in\R^3$ and $(1,1,1)\in\R^3$, respectively.\\
\quad\\
\textbf{Can we say $A=B$?} Since $(0,0,0)\in A$ but $(0,0,0)\notin B$, we have that $A\neq B$.\\
\textbf{Can we say $A=B$ almost everywhere?} We know the answer is NO.\\
\quad\\
However, both $A$ and $B$ are unitary balls in $\R^3$, so there exists an isometry $I:\R^3\to\R^3$ such that $I(A)=B$. Then, we can say they are equal under some isometry.


\end{frame}


\setbeamercolor{background canvas}{bg=white!10}
\begin{frame}\frametitle{Definitions}

Let $A,B\subset\R^3$ two sets describing 3D bodies.
\begin{definition}
An equivalent class of $A$ contains all the sets $X\subset\R^3$ for with exists $I:\R^3\to\R^3$ isometry such that $I(X)=A$ almost everywhere w.r.t. the Lebesgue measure.
\end{definition}
Then, we have that $A\sim B$ if and only if $\exists I:\R^3\to\R^3$ s.t. $A=I(B)$ a.e. We use $\mathcal{A}$ to denote the equivalent class of $A$.\\
\quad\\
Let $\mathcal{A}$ and $\mathcal{B}$ two non-necessarily equal equivalent classes as described before, and let $A\in\mathcal{A}$ and $B\in\mathcal{B}$ some elements from that classes. The next inequalities are trivial:
\begin{itemize}
\item $A\cup B\supseteq A$ and $A\cup B\supseteq B$. We reach qualities if and only if $A=B$.
\item $A\cap B\subseteq A$ and $A\cap B\subseteq B$. We reach qualities if and only if $A=B$.
\end{itemize}

\end{frame}


\setbeamercolor{background canvas}{bg=white!10}
\begin{frame}\frametitle{Definitions}

$A\cup B\supseteq A\cap B\implies \mu(A\cup B)\geq\mu(A\cap B)\implies\mu(A\cup B)-\mu(A\cap B)\geq0$, and we reach the equality if and only if $A=B$ almost everywhere. Then we can define our metric:

\begin{definition}
Let $\mathcal{A}$ and $\mathcal{B}$ two equivalent classes over isometries, and let $A$ and $B$ arbitrary elements from that classes, we define the metric $d(\mathcal{A},\mathcal{B})$ as
\begin{equation}
d(\mathcal{A},\mathcal{B})=\min_{A\in\mathcal{A},B\in\mathcal{B}}\left[\mu(A\cup B)-\mu(A\cap B)\right].
\end{equation}
\end{definition}
Notice that, $d(\mathcal{A},\mathcal{B})=0$ if and only if, $\mathcal{A}=\mathcal{B}$ (which implies that, for each pair $A$ and $B$, there exists an isometry $I:\R^3\to\R^3$ s.t. $I(A)=B$ almost everywhere).

\end{frame}


%\setbeamercolor{background canvas}{bg=white!10}
%\begin{frame}\frametitle{Title}
%
%\begin{columns}[c]
%
%\column{.5\textwidth}
%
%\column{.5\textwidth}
%\begin{figure}[ht!]
%\centering
%\includegraphics[width=0.4\textwidth]{Figures/20190809_1.png}
%\end{figure}
%
%\end{columns}
%
%\end{frame}

\end{document}